\chapter{Capítulo de prueba}
\lipsum[1-3]

\section{Sección de prueba}

En esta sección se probarán los entornos de los teoremas
\subsection{Entornos con el estilo \textit{plain}}

\begin{theorem}[Teorema choro]
hola soy un teorema.
\end{theorem}

\begin{lemma}[Lema Choro]
HOLA soy un lema.
\end{lemma}

\begin{corollary}[HOLA soy un corolario]
soy un corolario.
\end{corollary}

\begin{proposition}[Proposición chora]
HoLa Soy una proposición.
\end{proposition}


\subsection{Entornos con el estilo \textit{definition}}

\begin{definition}[EDO no lineal]
una \textit{EDO no lineal} es simplemente una EDO que no es lineal.
\end{definition}

\begin{exercise}[Ejercicio]
Lo anterior es tan trivial que ni siquiera vale como ejercicio.
\end{exercise}

\begin{example}[Inserte un ejemplo aquí]
Hola soy un ejemplo.
\end{example}



\subsection{Entornos con el estilo \textit{remark}}

\begin{remark}[Observación]
Estoy observando el teclado.
\end{remark}

\begin{note}[Nota importante]
Comprar más leche de soja.
\end{note}
\begin{conclusion}[Conclusión final]
En conclusión, me gusta el helado.
\end{conclusion}

\lipsum[4-7]
\subsection{Subsección de prueba}
\lipsum[8]
\subsection{Otra subsección de prueba}
\lipsum[9-20]
\chapter{Otro capítulo de prueba}
\lipsum[10]
\section{Hola soy otra sección de prueba}
\lipsum[11-40]